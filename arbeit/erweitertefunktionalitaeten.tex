\chapter{Erweiterte Funktionen}
In diesem Kapitel werden nun Funktionen beschrieben, die über die Grundausstattung der Web-App hinaus gehen, um diese zu erweitern und produktiver zu gestalten. 

\section{Statistiken}
Da eine Auswertung der eingegebenen Daten für Veranstaltungen unabdinglich ist, wurde ein weiterer Controller implementiert, welcher sämtliche speziellen Felder der Events aus der Datenbank abfragt und diesen dann die Werte der einzelnen Benutzer zuweist.\par

Im View wird dann eine grafische Auswertung gestartet, die mit Hilfe von Google Charts ansehnliche Graphen generiert, wo es Sinn ergibt und vergleichbare Werte von den Benutzern hinterlegt wurden. Außerdem gibt es allgemeine Statistiken, die die Veranstaltungen untereinander vergleichen und man so einen schnellen Überblick über die angelegten Events erhält.

\section{Lokalisierung von Clients}
Um die eingetragenen Helfer in dieser Webanwendung besser koordinieren zu können, wurde ein Modul implementiert, welches im Hintergrund der Web-App läuft und die aktuelle Position des Endgeräts über eine SSL verschlüsselte Verbindung an den Server übermittelt. Unter der Voraussetzung, dass der Client diesem Vorgang zustimmt, sind dem Server damit die Positionen der eingeloggten Benutzer bekannt. Diese Positionsdaten können dann von der Anwendung ausgewertet und in einer Karte von Google Maps angezeigt werden.\par

Jeder Client kann auf diese Weise die Positionen der anderen Helfer sehen. Der Vorteil liegt klar auf der Hand: eine zentral eingerichtete Verwaltung kann mit einem Blick sehen, wer sich an welcher Stelle auf dem Gelände befindet. So können Wege optimiert und gezielt Aufgaben verteilt werden, da ortsnahe Helfer die entsprechenden Aufgaben übernehmen können. Die kurzen Wege sorgen dann dafür, dass eine höhere Nutzung der Ressourcen (hier: die Helfer) möglich ist.\par

Eine wichtige Anforderung an diese Funktion ist, dass die Daten schnell ausgetauscht werden. Wenn zwischen den Updates der Positionen zu viel Zeit vergeht, ist die aktuelle Position nicht mehr aktuell und damit nicht mehr relevant.\par

Wie findet der Austausch der Positionen denn nun statt? Da es sich hier um sensible Daten handelt, ist eine sichere Übertragung Grundvoraussetzung. Allerdings steht im W3C Working Draft von HTML5, dass keine sichere, verschlüsselte Peer to Peer Verbindung mit HTML5 möglich ist \cite{w3cworkingdraft}. Im aktuellen Draft wurde auch keine sichere Verbindung definiert und es ist auch bisher keine vorgesehen \cite{w3ccurrent}.\\
Also liegt nahe weitere Techniken zu betrachten, welche einen sicheren Austausch von Daten zwischen Clients über einen Server ermöglichen.

\section{PUSH-Benachrichtigungen}
Bei Webanwendungen gibt es noch weitere Einschränkungen. So konnte für diese Arbeit nicht auf die systemeigenen Benachrichtigungsmechanismen zurückgegriffen werden, wie man sie aus nativen Applikationen her kennt. In den aktuellen Versionen aller mobilen Betriebssystemen sind Bereiche für Benachrichtigungen aus den jeweiligen Anwendungen implementiert um dem Benutzer zu signalisieren, dass neue Nachrichten vorliegen. Allerdings darf eine Web-App darauf nicht zugreifen, daher wurden für diese Anwendung eigene Methoden auf Basis von \emph{noty} \cite{noty}, einem jQuery Plugin für Benachrichtigungen, implementiert. Mit dieser Bibliothek wurde eine Benachrichtigungsleiste entwickelt, die am unteren Bildschirmrand Informationen anzeigen kann, wie zum Beispiel die oben angesprochene publish-Benachrichtigung vom WebSocket Server.\par

Auf diese Weise können nun auf allen Plattformen, die JavaScript aktiviert haben, Push-Benachrichtigungen angezeigt werden, wenn relevante Informationen über die WebSockets an das Endgerät gelangen.














