\chapter{Einleitung}
Jede große Veranstaltung muss mit viel Aufwand geplant und organisiert werden. Viele Hände sind dafür nötig, doch wie kann man diese Hände sinnvoll verwenden und ihre Arbeit miteinander verknüpfen?\par

Seit vielen Jahren bin ich nun schon Teil der bündischen Jugendbewegung im Deutschen Pfadfinderbund. Traditionsbewusst erinnert man sich an wichtige Ereignisse, wie der wohl bedeutendsten Verkündung der Jugendlichen in Deutschland im Jahr 1913 auf dem Hohen Meißner (ein 753,6 m über Normalnull hoher Berg in Nordhessen). Bei diesem Treffen entstand eine Formel, die die Entstehung deutscher bündischer Pfadfinder ausgelöst hat - Die Meißner Formel:
\begin{quote}
	\textit{\glqq Die Freideutsche Jugend will nach eigener Bestimmung, vor eigener Verantwortung, in innerer Wahrhaftigkeit ihr Leben gestalten. Für diese innere Freiheit tritt Sie unter allen Umständen geschlossen ein.\grqq{}}\cite[S. 109]{meissnerformel}
\end{quote}
Seitdem sind 100 Jahre vergangen, die Formel behält ihre Aktualität und es wird im Oktober eine große Veranstaltung mit über 4000 Pfadfindern und 90 freiwilligen Helfern an eben jenem Ort von 1913 stattfinden.\par

Durch diese Veranstaltung angeregt ist mit dieser Bachelorarbeit eine Webanwendung entstanden, welche sich auf die Organisation großer Veranstaltungen mit vielen Helfern spezialisiert.

* Was passiert in den einzelnen Kapiteln?

\section{Ziel der Arbeit}
Als Ziel soll eine Webanwendung entwickelt werden, die auf nahezu allen modernen Endgeräten ungeachtet des benutzten Betriebssystems lauffähig ist. Sie muss dabei grundlegende Funktionen zur Erstellung und Verwaltung von Veranstaltungen und deren zugehörigen Teilnehmern besitzen. Diese Einträge sollen zentral in einer Datenbank gespeichert werden und auf diese Datenbank sollen die zugeordneten Helfer entsprechend ihren Rechten Daten einsehen und verändern können.\par
Diese App wird im Oktober 2013 bei der oben beschriebenen Veranstaltung getestet und verwendet werden. Die dafür nötige Infrastruktur ist vor Ort gegeben.