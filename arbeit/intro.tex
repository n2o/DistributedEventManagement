\chapter{Einleitung}
Jede große Veranstaltung muss mit viel Aufwand geplant und organisiert werden. Viele Hände sind dafür nötig, doch wie kann man diese Hände sinnvoll verwenden und ihre Arbeit koordinieren?\par

Seit vielen Jahren bin ich Teil der bündischen Jugendbewegung im Deutschen Pfadfinderbund. Traditionsbewusst erinnert man sich an wichtige Ereignisse, wie der wohl bedeutendsten Verkündung der Jugendlichen in Deutschland im Jahr 1913 auf dem Hohen Meißner, einem 753,6 m über Normalnull hohen Berg in Nordhessen. Bei diesem Treffen entstand eine Formel, die die Entstehung deutscher bündischer Pfadfinder ausgelöst hat - die Meißner Formel:
\begin{quote}
	\textit{\glqq Die Freideutsche Jugend will nach eigener Bestimmung, vor eigener Verantwortung, in innerer Wahrhaftigkeit ihr Leben gestalten. Für diese innere Freiheit tritt Sie unter allen Umständen geschlossen ein.\grqq{}}\cite[S. 109]{meissnerformel}
\end{quote}
Seitdem sind 100 Jahre vergangen. Die Formel behält ihre Aktualität und es wird im Oktober das große Meißnerlager mit über 4.000 Pfadfindern und 90 freiwilligen Helfern an eben jenem Ort von 1913 stattfinden.

\section{Ziele der Arbeit}
Als Ziel soll eine Webanwendung entwickelt werden, die auf nahezu allen modernen Endgeräten ungeachtet des benutzten Betriebssystems lauffähig ist. Sie muss dabei grundlegende Funktionen zur Erstellung und Verwaltung von Veranstaltungen und deren zugehörigen Teilnehmern besitzen. Diese Einträge sollen zentral in einer MySQL-Datenbank gespeichert werden und auf diese Datenbank sollen die zugeordneten Helfer entsprechend ihren Rechten Daten einsehen und verändern können.\\
Die Meißner App soll eine mobile Seite bereitstellen, die für touchfähige Geräte optimiert ist und somit die bequeme Nutzung von Smartphones und Tablets ermöglicht.\\
Das Meißnerlager findet unter freiem Himmel statt, sodass die App für den Außenbereich und die mobile Nutzung optimiert sein muss. Auf knapp 40.000 $m^2$ Lagerplatzfläche werden sich ca. 90 freiwillige Helfer zwischen den ca. 4.000 erwarteten Teilnehmern bewegen. Dabei werden sie eigene Smartphones mit zu der Veranstaltung nehmen und über ihren Zugang zu der Meißner App ihre Aufgaben organisieren können. Damit die App auch bei schlechtem Empfang auf diesen Smartphones funktioniert, muss die Anwendung kritische Dateien cachen und lokal auf dem Gerät speichern.\par

Die Organisation durch diese Anwendung wird so einfach gehalten, dass sie von ungelernten Helfern produktiv genutzt werden kann. Mit einer einfachen Implementierung wird die App auch auf einem leistungsschwachem Server mit ausreichender Geschwindigkeit ausführbar sein, sodass keine kostenintensiven Computer notwendig sind.\par

Diese App wird im Oktober 2013 bei der oben beschriebenen Veranstaltung getestet und verwendet. Die dafür nötige Infrastruktur wird durch ein großes Telekommunikationsunternehmen gefördert. Sollte es zu Empfangsproblemen über die interne Antenne kommen, können sich die Smartphones in die installierten LTE Router einwählen.\par

Über die Homepage \url{http://www.meissner-2013.de} kann man sich über ein hinterlegtes PDF Formular zu der Veranstaltung anmelden. Dort werden viele Details abgefragt, welche den Großteil der Daten in der Datenbank ausmachen. Eine PDF wird verwendet, da die Pfadfinder oft klassisch veranlagt sind und die Anmeldung lieber ausdrucken und von Hand ausfüllen.

\section{Gliederung}
Das zweite Kapitel befasst sich mit der Implementierung der Grundfunktionen einer Webanwendung, Kapitel drei und vier erläutern Erweiterungen, die über die Grundaustattung der App hinausgehen.\\
Kapitel fünf befasst sich mit Benchmarks des Servers, um eine Vorstellung der Skalierbarkeit zu erhalten. Im sechsten Kapitel wurden ähnliche Arbeiten gesucht und die Unterschiede kurz verdeutlicht. Das siebte und damit letzte Kapitel fasst die Ergebnisse dieser Arbeit kurz zusammen und gibt einen Ausblick für potentielle Erweiterungen. 