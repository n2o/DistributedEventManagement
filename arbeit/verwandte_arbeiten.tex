\chapter{Verwandte Arbeiten}

\section{Verschiedene WordPress Plugins}
\paragraph{EventsEspresso}
Es handelt sich um ein WordPress Plugin, welches ein einfaches Veranstaltungsmanagement ermöglicht \cite{eventespresso}. Hierbei können in der Grundausstattung auch verschiedene Veranstaltungen, Teilnehmer und Helfer erstellt und verwaltet werden. Weitere Features wie die Verwaltung von Zahlungen, Erstellung von Eintrittskarten und verschiedene Sprachen gehören zum Basispaket.\\
An dieser Stelle endet auch schon die kostenfreie Version dieses Plugins und der Hersteller verlangt für weitere Premium Features hohe Preise: Für \$89.95 wird lediglich eine JSON API, eine Anbindung an soziale Netzwerke und ein Kalender integriert. Um den vollen Funktionsumfang zu haben und selbst Änderungen am Plugin vornehmen zu können, sind \$499.95 notwendig.\par

Features, wie Echtzeitaktualisierung, Chats oder die Optimierung für den Außenbereich wie in der Meißner App, gibt es nicht. Der Funktionsumfang der kostenlosen Version ist sehr spärlich und lässt sich nicht ohne Weiteres erweitern. Außerdem ist man an WordPress gebunden, da es nicht eigenständig lauffähig ist und eine angepasste Version für mobile Endgeräte gibt es auch nicht.

\paragraph{EventsManager}
Diese Software bewirbt sich selbst als das beliebteste WordPress Plugin für dieses Anwendungsgebiet \cite{eventsmanager}. Der Funktionsumfang ist ähnlich wie bei der oben beschriebenen Software. Möchte man auch hier weitere Addons nutzen, so müssen \$75 bezahlt werden.\par

Eine Unterstützung für großflächige Veranstaltungen im Außenbereich ist auch hier nicht vorgesehen, das Google Maps Modul dient lediglich zur Lokalisierung des eigenen Events und nicht zur Ortung der Helfer. Auch ist hier keine mobile Seite für Smartphone oder Tablets vorgesehen, es wird lediglich die gleiche Seite angezeigt wie an einem Desktop und ist nicht touchoptimiert. Bei einem Test dieses Plugin mit einem 7-Zoll Tablet aufzurufen, wurde die schlechte Anpassung an mobile Geräts dadurch deutlich, dass manche Container über anderen liegen und damit deren Inhalte verdecken.

%%%%%%%%%%%%%%%%%%%%%%%%%%%%%%%%%%%%%%%%%%%%%%%%%%

\section{Übersicht}
Bei der Recherche konnte keine Anwendung gefunden werden, welche als Plugin oder eigenständiges System die Organisation von Veranstaltungen unterstützt und dabei noch moderne Featues wie die Echtzeitaktualisierung, Publish / Subscribe oder Lokalisierung von Endgeräten bietet. Die Meißner App ist zudem dafür ausgelegt über eine große Fläche (auch unter freiem Himmel) und mit touchfähigen Geräten zu funktionieren. 