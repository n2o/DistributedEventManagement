\chapter{Zusammenfassung}
Mit dieser Bachelorarbeit ist die \emph{Meißner App} entstanden, welche das Erstellen von beliebig vielen Veranstaltungen und Benutzern ermöglicht. Sie bietet die Möglichkeit eigene Felder und Formulare für spezifische Events zu definieren und diese frei mit Werten der Benutzer zu füllen.\par

Die Anpassung an touchfähige Geräte ermöglicht eine einfache Bedienung mit Smartphone und Tablets, wobei ganz klar diese mobilen Endgeräte als Hauptzielgruppe zu verstehen sind.\par



%%%%%%%%%%%%%%%%%%%%%%%%%%%%%%%%%%%%%%%%%%%%%%%%%%

\section{Ausblick}
Diese Erweiterungen könnten noch implementiert werden, um den Funktionsumfang sinnvoll zu erweitern. Aus Zeitgründen konnten diese Funktionen nicht mit in die Bachelorarbeit aufgenommen und implementiert werden:

\paragraph{Geolocations}
Aktuell kann immer nur ein Endgerät eines Accounts auf der Google Maps Karte angezeigt werden. Zwar wird das Verbinden mehrerer Endgeräte über den gleichen Account von der Webanwendung unterstützt, allerdings wird auf der Karte immer nur ein Marker pro Account angezeigt. Das stört nicht, wenn beispielsweise ein Notebook und ein mobiles Endgerät über den gleichen Account eingeloggt sind, da das Notebook initial seine Position verschickt, diese aber nicht updatet, da sie sich nicht verändert (sofern es kein GPS besitzt). Das mobile Endgerät wird öfter seine Position updaten und damit aktuell auf der Karte angezeigt.

\paragraph{Chat}
Im Moment ist nur ein großes Channel zum chatten verfügbar. Das könnte man ändern und für jede Veranstaltung einen eigenen kleinen Channel erstellen.

\paragraph{Erweiterung von Publish / Subscribe}
Beliebige Veranstaltungen abonnieren, um damit mehr Informationen zu Veranstaltungen zu erhalten, die den Benutzer interessieren.

\paragraph{Kompression der übertragenen Daten}
Damit der Datentransfer möglichst gering gehalten wird, könnten die für die Webanwendung nötigen Daten weiter komprimiert werden. Dadurch werden Ladezeiten optimiert und das benötigte Datenvolumen minimiert.

\paragraph{Voice Chat}
Um direkten Kontakt zu anderen Helfern aufnehmen zu können, könnte man aus der App heraus einen Sprachanruf starten. So muss man nicht umständlich die Telefonnummer desjenigen Helfers suchen, sondern kann per Klick / Tippen auf den Namen einen Anruf starten.



