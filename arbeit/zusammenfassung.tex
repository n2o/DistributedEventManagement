\chapter{Zusammenfassung}
Mit dieser Bachelorarbeit ist die \emph{Meißner App} entstanden, welche das Erstellen von beliebig vielen Veranstaltungen und Benutzern ermöglicht. Sie bietet die Möglichkeit eigene Felder und Formulare für spezifische Events zu definieren und diese frei mit Werten der Benutzer zu füllen. Diese Werte stammen hauptsächlich aus den Anmeldungen der Teilnehmer.\par

Die Anpassung an touchfähige Geräte ermöglicht eine einfache Bedienung mit Smartphone und Tablets, wobei hauptsächlich mobile Endgeräte in Verwendung sein werden. Hierbei wurde auf die Unterstützung von Android 2.3 - 4.3 und iOS 5 - 7 Golden Master sowie die Browser Safari 6, Firefox 23 und Google Chrome 29 besonderen Wert gelegt.\\
Durch die einfache Bedienung können ungeschulte Personen die Webanwendung benutzen, verwalten und anpassen.\par

Moderne Webtechnologien aus dem Web 2.0 kommen hier zum Einsatz. Davon sind die bedeutensten das experimentelle HTML5 und die damit definierten WebSockets.

\paragraph{Besonderheiten} 
Für den produktiveren Einsatz der Helfer wurde eine Echtzeitaktualisierung mit Hilfe von HTML5 und WebSockets implementiert. Dadurch sind Erweiterungen wie die Geolokalisierung von Benutzern, PUSH-Benachrichtigungen innerhalb der Webanwendung und Chats für interessierte Benutzer möglich.\\
Optimiert ist die Anwendung für die Verwendung unter freiem Himmel, denn nur dort kann die Lokalisierung mit Smartphones genau erfolgen. Eine automatische statistische Auswertung der einzelnen Veranstaltungen ermöglichen eine grafisch ansprechende und verständliche Aufbereitung der eingegebenen Daten.\par

Mit diesen sinnvollen Erweiterungen erreicht die Anwendung einen Funktionsumfang, der bei keiner ähnlichen Anwendung gefunden werden konnte. Professionelle Entwickler verlangen viel Geld für ihre Produkte, während hier sowohl notwendige als auch erweiternde Funktionen für ein produktives Veranstaltungsmanagement-System gratis und Open Source zur Verfügung gestellt werden.\par

Zuletzt bleibt noch zu erwähnen, dass die Meißner App ein eigenständiges Programm ist, welches auf einem einfachen Webserver installiert werden kann. Sie benötigt kein besonderes Backend, sondern eine Standardumgebung, wie sie für heutige Webseiten üblich ist. Das heißt eine Apache Installation, MySQL Datenbank sowie die Unterstützung von HTML5 und PHP5.\\
Für die erweiterten Funktionen muss serverseitig node.js installiert sein.


%%%%%%%%%%%%%%%%%%%%%%%%%%%%%%%%%%%%%%%%%%%%%%%%%%

\section{Ausblick}
Die folgenden Erweiterungen könnten noch implementiert werden, um den Funktionsumfang sinnvoll zu erweitern. Aus Zeitgründen konnten diese Funktionen nicht mit in die Bachelorarbeit aufgenommen und implementiert werden:

\paragraph{Geolocations}
Unterstützung von mehreren Endgeräten, die mit dem gleichen Account eingeloggt sind.

\paragraph{Chat}
Im Moment ist nur ein Kanal zum chatten verfügbar. Das könnte man ändern und für jede Veranstaltung einen eigenen kleinen Channel erstellen.

\paragraph{Erweiterung von Publish / Subscribe}
Beliebige Veranstaltungen sollten abonniert werden können, um damit mehr Informationen zu Veranstaltungen zu erhalten, die den Benutzer interessieren.

\paragraph{Kompression der übertragenen Daten}
Um den Datentransfer möglichst gering zu halten, könnten die für die Webanwendung notwendigen Daten weiter komprimiert werden. Dadurch werden Ladezeiten optimiert und das benötigte Datenvolumen minimiert.

\paragraph{Voice Chat}
Um direkten Kontakt zu anderen Helfern aufnehmen zu können, könnte man aus der App heraus einen Sprachanruf starten. So muss man nicht umständlich die Telefonnummer desjenigen Helfers suchen, sondern kann per Klick / Tipp auf den Namen einen Anruf starten.



