\section{Analyse und Auswertung}

\begin{frame}{Benchmark WebServer}
	\begin{figure}
		\includegraphics<1->[width=0.7\textwidth]{fig/statistiken_ab.pdf}
	\end{figure}
	\begin{itemize}
		\item<2-> Apache Bench: 1000 Verbindungen, immer 10 gleichzeitig
		\item<3-> Praxis: Weniger Anfragen nötig durch Offline Cache
	\end{itemize}
\end{frame}

\begin{frame}{WebSocket Server}{Netzwerkauslastung}
	\begin{itemize}
		\item Nur 2 Bytes Overhead!
		\item Normale HTTP Anfragen (Polling o.Ä.): 700-800 Bytes Overhead
	\end{itemize}
\end{frame}

\begin{frame}{Besonderheiten der Webanwendung}
	\begin{itemize}
		\item<2-> Alleinstellung durch eigenständig lauffähige Webanwendung
		\item<3-> Unterstützung von mobilen Geräten
		\begin{itemize}
			\item <3->Offline Cache!
		\end{itemize}
		\item<4-> Automatisches Installationsskript für debianbasierte Systeme
		\item<5-> Für die Nutzung im Freien ausgelegt
		\item<6-> Modular aufgebaut
		\item<7-> Gratis!
		\begin{itemize}
			\item <7->Konkurrenzprodukte kosten mehrere hundert Dollar und sind nur als Erweiterung für WordPress verfügbar
		\end{itemize}
	\end{itemize}
\end{frame}