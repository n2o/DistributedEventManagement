\section{Analyse und Auswertung}

\begin{frame}{Benchmark WebServer}
	\begin{figure}
		\includegraphics<1->[width=0.7\textwidth]{fig/statistiken_ab.pdf}
	\end{figure}
	\begin{itemize}
		\item<2-> Apache Bench: 1000 Verbindungen, immer 10 gleichzeitig
		\item<3-> Praxis: Weniger Anfragen nötig durch Offline Cache
	\end{itemize}
\end{frame}

\begin{frame}{WebSocket Server}{Netzwerkauslastung}
	\begin{itemize}
		\item WebSockets: nur 2 Bytes Overhead!\pause
		\item Normale HTTP Anfragen (Polling o.Ä.): 700-800 Bytes Overhead\pause
		\item Besonders relevant bei hoher Anzahl von Clients, bspw. 1.000 Clients:
		\item[] \pause
	\end{itemize}

	\renewcommand{\arraystretch}{1.4}
	\centering
	\begin{tabular}{c|c|c}
		\textbf{Polling} & \textbf{WebSockets} & Einheit\\
		\hline
		800.000 &  2.000 & $\frac{Bytes}{Sekunde}$\\
		\hline
		6.400.000 &  16.000 & $\frac{Bits}{Sekunde}$\\
		\hline
		{\color{red}\textbf{6,104} :-(} & {\color{green}\textbf{0,015} :-)} & $\frac{MBit}{Sekunde}$\\
	\end{tabular}\pause

	\begin{itemize}
		\item[]
		\item[$\Rightarrow$] {\color{red}\textbf{Ersparnis von 400\% Traffic}}
	\end{itemize}

	\end{frame}

\begin{frame}{Besonderheiten der Webanwendung}
	\begin{itemize}
		\item<1-> Alleinstellung durch eigenständig lauffähige Webanwendung
		\item<2-> Unterstützung von mobilen Geräten
		\begin{itemize}
			\item <2->Offline Cache!
		\end{itemize}
		\item<3-> Automatisches Installationsskript für debianbasierte Systeme
		\item<4-> Für die Nutzung im Freien ausgelegt
		\item<5-> Modular aufgebaut
		\item<6-> Gratis!
		\begin{itemize}
			\item <6->Konkurrenzprodukte kosten mehrere hundert Dollar und sind nur als Erweiterung für WordPress verfügbar
		\end{itemize}
	\end{itemize}
\end{frame}