\section{Erweiterungen}

\begin{frame}{WebSockets}
	\begin{figure}
		\includegraphics<1>[width=0.9\textwidth]{fig/websockets_beamer_3.pdf}
		\includegraphics<2>[width=0.9\textwidth]{fig/websockets_beamer_4.pdf}
		\includegraphics<3>[width=0.9\textwidth]{fig/websockets_beamer_5.pdf}
		\includegraphics<4>[width=0.9\textwidth]{fig/websockets_beamer.pdf}
	\end{figure}
	\begin{itemize}
		\item<1-> TCP-basiertes Netzwerkprotokoll
		\item<1-> bidirektionale Verbindung zwischen Client und Webserver
	\end{itemize}
\end{frame}

\begin{frame}{Authentifizierung beim WebSocket Server}{Public Key Kryptographie}
	\begin{figure}
		\includegraphics<1>[width=0.9\textwidth]{fig/websockets_auth_1.pdf}
		\includegraphics<2>[width=0.9\textwidth]{fig/websockets_auth_3.pdf}
		\includegraphics<3>[width=0.9\textwidth]{fig/websockets_auth_4.pdf}
		\includegraphics<4>[width=0.9\textwidth]{fig/websockets_auth_5.pdf}
		\includegraphics<5>[width=0.9\textwidth]{fig/websockets_auth_6.pdf}
		\includegraphics<6>[width=0.9\textwidth]{fig/websockets_auth.pdf}
	\end{figure}
\end{frame}

\begin{frame}{Publish / Subscribe}
	\begin{itemize}
		\item Client abonniert eine Veranstaltung oder eine Seite
		\begin{itemize}
			\item[$\Rightarrow$] Ermöglicht automatische Updatebenachrichtigung, Chats, usw. 
		\end{itemize}
		\item Client und Server müssen nichts voneinander wissen
		\item Publish funktioniert ohne JavaScript!
		\begin{itemize}
			\item Direkte Kommunikation zwischen Apache und WebSocket Server
		\end{itemize}
	\end{itemize}

	\begin{itemize}
		\item[]<2-> Wird hier genutzt für:
		\begin{enumerate}
			\item<2-> Benachrichtigung bei Änderung einer Veranstaltung
			\item<2-> Chats
			\item<2-> Geolokalisierung
		\end{enumerate}
	\end{itemize}
\end{frame}

\begin{frame}{Chats}
	\begin{figure}
		\includegraphics<1>[width=0.9\textwidth]{fig/chat_3.pdf}
		\includegraphics<2>[width=0.9\textwidth]{fig/chat_4.pdf}
		\includegraphics<3>[width=0.9\textwidth]{fig/chat_5.pdf}
		\includegraphics<4>[width=0.9\textwidth]{fig/chat.pdf}
	\end{figure}
\end{frame}

\begin{frame}{Geolocations}
	\begin{figure}
		\includegraphics<1>[width=0.7\textwidth]{fig/geolocations_beamer_1.pdf}
		\includegraphics<2>[width=0.7\textwidth]{fig/geolocations_beamer_2.pdf}
		\includegraphics<3>[width=0.7\textwidth]{fig/geolocations_beamer_3.pdf}
		\includegraphics<4>[width=0.7\textwidth]{fig/geolocations_beamer_4.pdf}
	\end{figure}
\end{frame}

\begin{frame}{Offline Cache}
	\begin{itemize}
		\item Automatisches Cachen von Bildern, Skripten, Seiten, \dots
		\item Bei Änderung der Datenbank: Update des Caches notwendig
	\end{itemize}
	\begin{itemize}
		\item[$\Rightarrow$] Deutlich schnellere Bedienung
		\item[$\Rightarrow$] Unempfindlicher für Funknetzabbrüche
	\end{itemize}
\end{frame}

% \begin{frame}[fragile]{Offline Cache: Manifest}
% 	\begin{verbatim}
% 		CACHE MANIFEST
		
% 		CACHE:
% 		/favicon.ico
% 		stylesheet.css
% 		images/logo.png
% 		scripts/main.js

% 		# Resources that require the user to be online.
% 		NETWORK:
% 		login.php
		
% 		# Version: 42
% 	\end{verbatim}
% \end{frame}