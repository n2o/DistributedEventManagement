\section{Erweiterungen}

\begin{frame}{Die Meißner App}{Erweiterung: WebSockets}
	\begin{figure}
		\includegraphics<1>[width=0.9\textwidth]{fig/websockets_beamer_3.pdf}
		\includegraphics<2>[width=0.9\textwidth]{fig/websockets_beamer_4.pdf}
		\includegraphics<3>[width=0.9\textwidth]{fig/websockets_beamer_5.pdf}
		\includegraphics<4>[width=0.9\textwidth]{fig/websockets_beamer.pdf}
	\end{figure}
\end{frame}

\begin{frame}{Die Meißner App}{Erweiterung: Authentifizierung beim WebSocket Server}
	\begin{figure}
		\includegraphics<1>[width=0.9\textwidth]{fig/websockets_auth_1.pdf}
		\includegraphics<2>[width=0.9\textwidth]{fig/websockets_auth_3.pdf}
		\includegraphics<3>[width=0.9\textwidth]{fig/websockets_auth_4.pdf}
		\includegraphics<4>[width=0.9\textwidth]{fig/websockets_auth_5.pdf}
		\includegraphics<5>[width=0.9\textwidth]{fig/websockets_auth_6.pdf}
		\includegraphics<6>[width=0.9\textwidth]{fig/websockets_auth.pdf}
	\end{figure}
\end{frame}

\begin{frame}{Die Meißner App}{Erweiterung: Publish / Subscribe}
	\begin{itemize}
		\item<1-> Client abonniert eine Veranstaltung oder eine Seite
		\begin{itemize}
			\item[$\Rightarrow$]<2-> Ermöglicht automatische Updatebenachrichtigung, Chats, usw. 
		\end{itemize}
		\item<3-> Client und Server müssen nichts voneinander wissen
		\item<4-> Ohne JavaScript!
		\begin{itemize}
			\item<5-> Direkte Kommunikation zwischen Apache und WebSocket Server
		\end{itemize}
	\end{itemize}

	\begin{itemize}
		\item<6-> Wird hier genutzt für:
		\begin{enumerate}
			\item<6-> Chats
			\item<6-> Geolokalisierung
			\item<6-> Benachrichtigung bei Änderung einer Veranstaltung
		\end{enumerate}
	\end{itemize}
\end{frame}

\begin{frame}{Die Meißner App}{Erweiterung: Chats}
	\begin{figure}
		\includegraphics<1>[width=0.9\textwidth]{fig/chat_3.pdf}
		\includegraphics<2>[width=0.9\textwidth]{fig/chat_4.pdf}
		\includegraphics<3>[width=0.9\textwidth]{fig/chat_5.pdf}
		\includegraphics<4>[width=0.9\textwidth]{fig/chat.pdf}
	\end{figure}
\end{frame}

\begin{frame}{Die Meißner App}{Erweiterung: Geolocations}
	\begin{figure}
		\includegraphics<1>[width=0.7\textwidth]{fig/geolocations_beamer_1.pdf}
		\includegraphics<2>[width=0.7\textwidth]{fig/geolocations_beamer_2.pdf}
		\includegraphics<3>[width=0.7\textwidth]{fig/geolocations_beamer_3.pdf}
		\includegraphics<4>[width=0.7\textwidth]{fig/geolocations_beamer_4.pdf}
	\end{figure}
\end{frame}

\begin{frame}{Besonderheit der Webanwendung}
	\begin{itemize}
		\item<2-> Alleinstellung durch eigenständig lauffähige Webanwendung
		\item<3-> Automatisches Installationsskript für debianbasierte Systeme
		\item<4-> Gratis!
		\begin{itemize}
			\item <4->Konkurrenzprodukte kosten mehrere Hundert Dollar und sind nur als Erweiterung für WordPress verfügbar
		\end{itemize}
		\item<5-> Frei gestaltbar
		\item<6-> Modular aufgebaut
	\end{itemize}
\end{frame}