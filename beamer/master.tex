\pdfoptionpdfminorversion=5
\documentclass[9pt,hyperref={pdfpagelabels=false}]{beamer}
\mode<presentation> {
    \usetheme{HHUD}
    \setbeamercovered{invisible}
}
\usepackage[ngerman]{babel}
\usepackage[utf8x]{inputenc}
\usepackage{times}
\usepackage[T1]{fontenc}
\usepackage{amsmath}
\usepackage{subfigure}
\usepackage{graphicx}
\usepackage{hyperref}
\usepackage{xmpmulti}
\usepackage{multicol}
\usepackage{appendixnumberbeamer}
\usepackage{tabularx}
\usepackage{listings}

% background image
\usebackgroundtemplate{\includegraphics[width=\paperwidth]{fig/background}}
% commands for low and high decoration in frame foot
\newcommand{\footdecorationlow}{\usebackgroundtemplate{\includegraphics[width=\paperwidth]{fig/background_small}}}
\newcommand{\footdecorationhigh}{\usebackgroundtemplate{\includegraphics[width=\paperwidth]{fig/background}}}

% Fix build errors on debian (http://bugs.debian.org/cgi-bin/bugreport.cgi?bug=452333)
\providecommand \thispdfpagelabel[1]{} {}

%% Die folgenden Zeilen können auskommentiert werden, um vor jedem Kapitel eine Gliederungsfolie einzufügen
% \AtBeginSection[] {
%   \footdecorationhigh
%   \begin{frame}<beamer>
%     \thispagestyle{empty}
%     \frametitle{Gliederung}
%     \vspace{-5mm}
%     \tableofcontents[currentsection]
%   \end{frame}
%   \footdecorationlow
% }

% % % % % % % % % %  CHANGE TOPIC AND AUTHOR INFORMATION HERE % % % % % % % % %
\newcommand{\abschluss}{Bachelor}                              % HIER UNZUTREFFENDES LÖSCHEN
\title{\abschluss{}arbeit:\\Verteiltes Veranstaltungsmanagement mit einer mobilen Webanwendung}                      % HIER DEN TITEL DER ARBEIT EINTRAGEN
\author{Christian Meter}                                                       % HIER DEN NAMEN UND VORNAMEN EINTRAGEN
\date{10.10.2013}                                                                % HIER DAS PRÄSENTATIONSDATUM EINTRAGEN
% % % % % % % % % % % % % % % % % % % % % % % % % % % % % % % % % % % % % % % %
\institute{Institut für Informatik\\Heinrich-Heine-Universität Düsseldorf}
\subject{Informatik}

%
% Hier beginnt das Dokument
%
\begin{document}

  \footdecorationhigh
  \begin{frame}
    \thispagestyle{empty}
    \titlepage
  \end{frame}

  \begin{frame}
    \thispagestyle{empty}
    \frametitle{Gliederung}
    \vspace{-5mm}
    \tableofcontents
  \end{frame}
  \footdecorationlow

  % % % % % % % % % % Ab hier werden die LaTeX-Dateien der einzelnen Abschnitte eingefügt % % % % % % % % % %

  \section{Einleitung}

\begin{frame}
	\frametitle{Problembeschreibung}
	\begin{itemize}
		\item<1-> 1 Veranstaltung
		\item<2-> 3500 Teilnehmer
		\item<3-> 89 ehrenamtliche Helfer, zunächst deutschlandweit verteilt
		\item<4-> Verschiedene Aufgabenverteilung
		\begin{enumerate}
			\item<4-> Anmeldungen bearbeiten
			\item<4-> Helfer verwalten
			\item<4-> Statistiken erstellen und auswerten
			\item<4-> \dots
		\end{enumerate}
	\end{itemize}

	\begin{block}{Wie soll man produktiv zusammenarbeiten?}<5->
		Viele Personen, verschiedene Standorte, Betriebssysteme, usw.
	\end{block}
\end{frame}

%%%%%%%%%%%%%%%%%%%%%%%%%%%%%%%%%%%%%%%%%%%%%%%%%%%%%%%%%%%%%%%%%%%%%%%%%%%%%%%%%%%%%%

\begin{frame}
	\frametitle{Was wird benötigt?}
	\begin{figure}
		\includegraphics<2>[width=0.8\textwidth]{fig/aufbau_app_1.pdf}
		\includegraphics<3>[width=0.8\textwidth]{fig/aufbau_app_2.pdf}
		\includegraphics<4>[width=0.8\textwidth]{fig/aufbau_app_3.pdf}
		\includegraphics<5>[width=0.8\textwidth]{fig/aufbau_app_4.pdf}
		\includegraphics<6>[width=0.8\textwidth]{fig/aufbau_app_5.pdf}
		\includegraphics<7>[width=0.8\textwidth]{fig/aufbau_app_6.pdf}
		\includegraphics<8>[width=0.8\textwidth]{fig/aufbau_app.pdf}
	\end{figure}
\end{frame}

  \section{Webanwendung}

\begin{frame}{Die Meißner App}{Grundfunktionen}
	\begin{figure}
		\includegraphics<1>[width=0.8\textwidth]{fig/grundfunktionen_2.pdf}
		\includegraphics<2>[width=0.8\textwidth]{fig/grundfunktionen_3.pdf}
		\includegraphics<3>[width=0.8\textwidth]{fig/grundfunktionen_4.pdf}
		\includegraphics<4>[width=0.8\textwidth]{fig/grundfunktionen.pdf}
	\end{figure}
\end{frame}

\begin{frame}{Die Meißner App}{Erweiterung: Statistiken}
	\begin{itemize}
		\item<1> Selbstständige Visualisierung!
	\end{itemize}

	\begin{figure}
		\includegraphics<2>[width=0.8\textwidth]{fig/statistiken_2.pdf}
		\includegraphics<3>[width=0.8\textwidth]{fig/statistiken_3.pdf}
		\includegraphics<4>[width=0.8\textwidth]{fig/statistiken.pdf}
	\end{figure}
\end{frame}

\begin{frame}{Die Meißner App}{Erweiterung: WebSockets}
	\begin{itemize}
		\item Was sind WebSockets?
		\item Hier die Statistik reinpacken, die zeigt, warum WebSockets toll sind (aus Bachelorarbeit)
	\end{itemize}
\end{frame}

\begin{frame}{Die Meißner App}{Erweiterung: Publish / Subscribe}
	\begin{itemize}
		\item Wie funktioniert das?
		\item Wofür wurde das eingebaut?
		\item Übergang zur Authentifizierung
	\end{itemize}
\end{frame}

\begin{frame}{Die Meißner App}{Erweiterung: Authentifizierung beim WebSocket Server}
	\begin{itemize}
		\item Wie funktioniert das?
		\item Public Key Verfahren
		\item Damit kommen wir zu Chats und Geolocations
	\end{itemize}
\end{frame}

\begin{frame}{Die Meißner App}{Erweiterung: Chats}
	\begin{figure}
		\includegraphics<1>[width=0.9\textwidth]{fig/chat_3.pdf}
		\includegraphics<2>[width=0.9\textwidth]{fig/chat_4.pdf}
		\includegraphics<3>[width=0.9\textwidth]{fig/chat_5.pdf}
		\includegraphics<4>[width=0.9\textwidth]{fig/chat.pdf}
	\end{figure}
\end{frame}

\begin{frame}{Die Meißner App}{Erweiterung: Geolocations}
	\begin{itemize}
		\item Jeder eingeloggte Benutzer schickt seine Position an den Server
		\item Über WebSockets findet der Austausch statt
		\item Jede Aktualisierung wird per WebSockets an die anderen per Broadcast verschickt
	\end{itemize}
\end{frame}

\begin{frame}{Besonderheit der Webanwendung}
	\begin{itemize}
		\item Was macht sie besonders?
		\item Alleinstellung durch eigenständig lauffähige Webanwendung
		\item OpenSource
		\item Konkurrenzprodukte kosten mehrere Hundert Dollar
	\end{itemize}
\end{frame}
  
  \section{Analyse und Auswertung}

\begin{frame}{Benchmark WebServer}
	Apache knechten
	\begin{itemize}
		\item Hier Grafik aus Bachelorarbeit, vereinfacht auf Standard App und 100 Eigenschaften
		\item Einbruch der Zugriffszeiten erst bei 100 Eigenschaften von einem Event
	\end{itemize}
\end{frame}

\begin{frame}{WebSocket Server}{Netzwerkauslastung}
	\begin{itemize}
		\item Vergleich mit Polling
		\item Rechnung einbauen die zeigt, dass WebSockets um ein Vielfaches kleiner sind
	\end{itemize}
\end{frame}

  \appendix

  % % % % % % % % % % Ende der eingefügten LaTeX-Dateien % % % % % % % % % %

\end{document}

%
% Hier endet das Dokument
%