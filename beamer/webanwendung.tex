\section{Webanwendung}

\begin{frame}{Die Meißner App}{Grundfunktionen}
	\begin{figure}
		\includegraphics<1>[width=0.8\textwidth]{fig/grundfunktionen_2.pdf}
		\includegraphics<2>[width=0.8\textwidth]{fig/grundfunktionen_3.pdf}
		\includegraphics<3>[width=0.8\textwidth]{fig/grundfunktionen_4.pdf}
		\includegraphics<4>[width=0.8\textwidth]{fig/grundfunktionen.pdf}
	\end{figure}
\end{frame}

\begin{frame}{Die Meißner App}{Erweiterung: Statistiken}
	\begin{itemize}
		\item<1> Selbstständige Visualisierung!
	\end{itemize}

	\begin{figure}
		\includegraphics<2>[width=0.8\textwidth]{fig/statistiken_2.pdf}
		\includegraphics<3>[width=0.8\textwidth]{fig/statistiken_3.pdf}
		\includegraphics<4>[width=0.8\textwidth]{fig/statistiken.pdf}
	\end{figure}
\end{frame}

\begin{frame}{Die Meißner App}{Erweiterung: WebSockets}
	\begin{itemize}
		\item Was sind WebSockets?
		\item Hier die Statistik reinpacken, die zeigt, warum WebSockets toll sind (aus Bachelorarbeit)
	\end{itemize}
\end{frame}

\begin{frame}{Die Meißner App}{Erweiterung: Publish / Subscribe}
	\begin{itemize}
		\item Wie funktioniert das?
		\item Wofür wurde das eingebaut?
		\item Übergang zur Authentifizierung
	\end{itemize}
\end{frame}

\begin{frame}{Die Meißner App}{Erweiterung: Authentifizierung beim WebSocket Server}
	\begin{itemize}
		\item Wie funktioniert das?
		\item Public Key Verfahren
		\item Damit kommen wir zu Chats und Geolocations
	\end{itemize}
\end{frame}

\begin{frame}{Die Meißner App}{Erweiterung: Chats}
	\begin{figure}
		\includegraphics<1>[width=0.9\textwidth]{fig/chat_3.pdf}
		\includegraphics<2>[width=0.9\textwidth]{fig/chat_4.pdf}
		\includegraphics<3>[width=0.9\textwidth]{fig/chat_5.pdf}
		\includegraphics<4>[width=0.9\textwidth]{fig/chat.pdf}
	\end{figure}
\end{frame}

\begin{frame}{Die Meißner App}{Erweiterung: Geolocations}
	\begin{itemize}
		\item Jeder eingeloggte Benutzer schickt seine Position an den Server
		\item Über WebSockets findet der Austausch statt
		\item Jede Aktualisierung wird per WebSockets an die anderen per Broadcast verschickt
	\end{itemize}
\end{frame}

\begin{frame}{Besonderheit der Webanwendung}
	\begin{itemize}
		\item Was macht sie besonders?
		\item Alleinstellung durch eigenständig lauffähige Webanwendung
		\item OpenSource
		\item Konkurrenzprodukte kosten mehrere Hundert Dollar
	\end{itemize}
\end{frame}