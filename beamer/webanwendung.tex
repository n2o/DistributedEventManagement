\section{Webanwendung}

\begin{frame}{Die Meißner App}{Grundfunktionen}
	\begin{figure}
		\includegraphics<1>[width=0.8\textwidth]{fig/grundfunktionen_2.pdf}
		\includegraphics<2>[width=0.8\textwidth]{fig/grundfunktionen_3.pdf}
		\includegraphics<3>[width=0.8\textwidth]{fig/grundfunktionen_4.pdf}
		\includegraphics<4>[width=0.8\textwidth]{fig/grundfunktionen.pdf}
	\end{figure}
\end{frame}

\begin{frame}{Die Meißner App}{Erweiterungen}
	Statistiken
	\begin{itemize}
		\item Selbstständige Visualisierung
		\item Vergleicht Werte miteinander, wenn mindestens zwei verschiedene Werte eingetragen wurden
	\end{itemize}
\end{frame}

\begin{frame}{Die Meißner App}{Erweiterungen}
	WebSockets
	\begin{itemize}
		\item Was sind WebSockets?
		\item Hier die Statistik reinpacken, die zeigt, warum WebSockets toll sind (aus Bachelorarbeit)
	\end{itemize}
\end{frame}

\begin{frame}{Die Meißner App}{Erweiterungen}
	Publish / Subscribe
	\begin{itemize}
		\item Wie funktioniert das?
		\item Wofür wurde das eingebaut?
		\item Übergang zur Authentifizierung
	\end{itemize}
\end{frame}

\begin{frame}{Die Meißner App}{Erweiterungen}
	Authentifizierung beim WebSocket Server
	\begin{itemize}
		\item Wie funktioniert das?
		\item Public Key Verfahren
		\item Damit kommen wir zu Chats und Geolocations
	\end{itemize}
\end{frame}

\begin{frame}{Die Meißner App}{Erweiterungen}
	Chats
	\begin{itemize}
		\item History wird auf dem Server gespeichert
		\item Jeder eingeloggte Benutzer kann die letzten 100 Nachrichten lesen und schreiben
		\item Ein großer Channel
	\end{itemize}
\end{frame}

\begin{frame}{Die Meißner App}{Erweiterungen}
	Geolocations
	\begin{itemize}
		\item Jeder eingeloggte Benutzer schickt seine Position an den Server
		\item Über WebSockets findet der Austausch statt
		\item Jede Aktualisierung wird per WebSockets an die anderen per Broadcast verschickt
	\end{itemize}
\end{frame}

\begin{frame}{Besonderheit der Webanwendung}
	\begin{itemize}
		\item Was macht sie besonders?
		\item Alleinstellung durch eigenständig lauffähige Webanwendung
		\item OpenSource
		\item Konkurrenzprodukte kosten mehrere Hundert Dollar
	\end{itemize}
\end{frame}